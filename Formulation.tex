
\section{Formulation}

	We will be using Markov Models to find the probability of the events.
Markov models are graphs of events, that have weighted edges with the probability of their occurrence.
This graph is then converted into a matrix where the values in row $n$ column $i$ represent the odds of moving from state $n$ to $i$.
This is a useful formulation because if you take the product of two of these matrices then it shows the odds of going from one state to another after 2 changes of state.
A proof, mostly to convince myself, is in Appendix I for the two by two case.

We use this to represent the probability to get certain outcome from a starting state when the outcome doesn't depend on path. For example to path to getting a Yahtzee doesn't matter, all that matters is the outcome, and the hopefully 50 points that come along with it.

Now rather than having to find the odds for each possibility and generalize those for n changes of state, we can find the odds of changing from each state to another and the generalsation to n state changes is done for us.

We also express the markov matrix as the product of two matrices, that we denote as $D$ and $P$. $D$ is a diagonal matrix that has the integer number of possible ways to move from the state of the row/column that it belongs to

\subsection{Graphical Example}

\begin{figure}
  \includegraphics[width=\linewidth]{image11.jpeg}
  \caption[3-State Markov]{Probability transition diagram for a 3-state Markov chain}
  \label{state}
\end{figure}


Because we can use a matrix to represent a directed graph, we can use this matrix
to apply a weight to the edits and predict the outcome of events. Figure \ref{state}
shows an example of a graph and the corresponding matrix.\footnote{http://www.intechopen.com/books/matlab-a-fundamental-tool-for-scientific-computing-and-engineering-applications-volume-2/wireless-channel-model-with-markov-chains-using-matlab}

To find the odds of a transition, $P$ can be multiplied by the column vector
$\mathbf{x}=({P_A, P_B, P_C})^T$, where $P_X$ is the probability that the system
is in state $X$. So, $P\mathbf{x}$ results in the probability after one set
of changes, and $P(P\mathbf{x})$ results in the probability after two state
changes. This can be simplified to $P(P\mathbf{x}) = (PP)\mathbf{x} = P^2\mathbf{x}$.
This can be further generalized to state that the probability after $n$ state
changes is: $P^n\mathbf{x}$. In the case of Yahtzee, we can use this to optimize
how we play so that we can consistently beat other players.
