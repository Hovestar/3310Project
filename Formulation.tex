\section{Formulation}
	
	We will be using Markov Models to find the probability of the events. 
Markov models are graphs of events, that have weighted edges with the probability of their occurrence. 
This graph is then converted into a matrix where the values in row $n$ column $i$ represent the odds of moving from state $n$ to $i$. 
This is a useful formulation because if you take the product of two of these matrices then it shows the odds of going from one state to another after 2 changes of state. 
A proof, mostly to convince myself, is in Appendix I for the two by two case.

We use this to represent the probability to get certain outcome from a starting state when the outcome doesn't depend on path. For example to path to getting a yatzee doesn't matter, all that matters is the outcome, and the hopefully 50 points that come along with it. 

Now rather than having to find the odds for each possibility and generalize those for n changes of state, we can find the odds of changing from each state to another and the generalsation to n state changes is done for us. 

We also express the markov matrix as the product of two matrices, that we denote as $D$ and $P$. $D$ is a diagonal matrix that has the integer number of possible ways to move from the state of the row/column that it belongs to
