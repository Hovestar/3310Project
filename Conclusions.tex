\section{Conclusions}

Our results show that the point system for a game of Yahtzee is not necessarily
based on the expected values for each of the possible scoring criterea.
For example, while the expected value of a Yahtzee is about 2.3 points, the expected
value for a full house (which the game rules give less points to than rolling
a Yahtzee) is about 9 points, slightly more than triple the expected value for a Yahtzee.
This shows that while the game's scoring system encourages a certain style of
play (namely, trying for Yahtzees), a much migher return can be expected from
rolling for different combinations (such as a full house).

The part of the game that really produces points, more than either the Yahtzee or the Full House is the top, which gives an expected value 2.1 times the number rolled for as an expected value, so going for sixes gives an expected value of almost 13 points, being the biggest bang for your buck in Yahtzee. 
