\section{Introduction}

The game of Yahtzee, where players roll six-sided dice multiple times, and
calculate their scores based on the values of the dice. As with many games
that utilize dice, propability and the ability to make predictions plays a great
part in generating a high score.

Markov Probability Models are state based conditional directed graphs that
allow predictions on systems where independence cannot be assumed. Graphs can
be represented as matrices, which are where Markov systems gain their power.
We would like to use these methods to analyze the game of Yahtzee in order to
predict the best strategies for fun (and profit).

\subsection{What is Yahtzee?}

Yahtzee is a game where players roll six-sided dice. After each roll, the
player has the option to roll any number of the five dice again (up to a total
of 3 rolls). Then, the values of the dice are evaluated and scored. For example,
points could be owned for two of a kind, or full house.

This process is continued for 13 rounds. However, the same categories for scoring
cannot be used more than once per game (set of 13 rounds). Finally, the scores
are totaled from each round, with the player with the highest score winning the
game.

\subsection{Using Markov Models to Analyze Yahtzee}

As Yahtzee revolves around rolling dice, players have an equal chance of
winning the game. Theoretically, victory should be based on luck and not
strategy.

Markov Probability Models are a good fit for analyzing Yahtzee, since each
die can be represented as having six states, ranging from one to six, and also
because each new roll has some transition, whether that be to the same state or
a new state. The probability of each type of set based on the scoring scale will
be used to compare our findings to the values provided by the game of Yahtzee.
Our goal is to examine how closely the point values match with the probability
of rolling a certain pattern of dice.
